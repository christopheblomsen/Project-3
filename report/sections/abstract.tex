This paper covers the numerical solutions of a Penning trap. The Penning trap
is a device that confine particles inside a device using electric and magnetic field.
 In this report, we start with deriving the equations of motion for charged
 particles confined in the Penning trap.
Before we build the simulation, we derive the general solution and find
upper and lower bounds on the displacement of the particle from the centre of the trap.
Next, we look at the specific case of Ca$^+$ ions and how they would behave in
said Penning trap with the configurations $B_0 = 1.00$T, $V_0 = 25.0$mV and
$d = 500\si{\micro\meter}$.
We accomplish this by using two different numerical methods, namely the Forward-Euler and the
Runge-Kutta method implemented with object-oriented programming.
The results are used to investigate the effect of the Coulomb interaction
between the particles in the trap. The possibility of energy exchange between the
particles allows the particles to spread out over a lager domain both in
real space and in phase space.
Since the particles display periodic motion we expect to find resonance phenomena.
We study a time dependent sinusoidal electric field with different amplitudes
and frequencies.
At $\omega_V \approx 2.2$ MHz we find that
most of the ions are escaping the trap during our simulation time of $500 \mu$s.
When we simulate the same system with Coulomb interactions we find that
this frequency is split up, and shifted to a slightly lower value.
The codes used for this project can be found at \url{}.
 


%% %Explain what a Penning trap is:
%% %revise: read again to check the continuity
%% %do we use all the results we have derived?
An illustration of a Penning trap is in \autoref{fig:penning}, it
consists of three electrodes, two caps and one ring.
The caps and the ring has opposite charges. When the caps are positive, as shown in \autoref{fig:penning}
they set up an $\bold{E}$-field given by
\begin{equation}
V(x, y, z) = \frac{V_0}{2d^2}{2z^2 - x^2 - y^2}.
\end{equation}

Where $V_0$ indicates the strength of the potential and $d$ is a measure of the size of the trap.
revise(details can be added here if we find it relevant)
This potential has an unstable equilibrium along a verticle line in the centre of the ring.
The particles we will be studying are positive we need to prevent them from being pulled to the negatively
charged ring electrode. To do this, we place the entire trap inside a magnet so that there is a constant
magnetic field within the trap

\begin{equation}
\bold{B} = B_0 \bold{e}_z.
\end{equation}

Since the Lorentz force on the particles is perpendicular to the magnetic field and their direction of
motion, a sufficiently strong magnetic field will force the particles to orbit around the centre of the trap
and not escape or hit the negative electrode.

%todo: Derive the equations of motion and stuff.
Newton's second law can find the equations of motion for a positive particle within the trap. The
force on the particle is the Lorentz force. Since we are only trapping
single Ca$^+$-ions, we can safely neglect gravity.

The Lorentz force is given by
\begin{equation}
\bold{F} = q\bold{E} + q\dot{\bold{r}} \times \bold{B},
\end{equation}

where $q$ is the particle's charge, $\bold{r}$ is the position vector and $\bold{B}$ is the magnetic field vector.

The gradient gives the electric field
of the electric potential
\begin{equation}
\bold{E} = -\nabla V = -\frac{V_0}{d^2}\left(2 \bold{e}_z - x\bold{e}_x - y\bold{e}_y\right),
\end{equation}

and the cross product of the velocity and the $\bold{B}$-field can be written as

\begin{equation}
\dot{\bold{r}} \times \bold{B} = B_0 \dot{y} \bold{e}_x - B_0 \dot{x} \bold{e}_y.
\end{equation}

Thus the equations of motion can be written as
\begin{align}
      \ddot{x} - \omega_{0}\dot{y} - \frac{1}{2}\omega_{z}^{2}x &= 0, \label{eq:eom_x}\\
  \ddot{y} + \omega_{0}\dot{x} - \frac{1}{2}\omega_{z}^{2}y &= 0, \label{eq:eom_y}\\
  \ddot{z} + \omega_{z}^{2}z &= 0, \label{eq:eom_z}
\end{align}

where we have defined $\omega_0$ and $\omega_z$ by
\begin{align}
\omega_0 \equiv& \frac{qB_0}{m}, \label{eq:omega_0} \\
\omega_z^2 \equiv& \frac{2qV_0}{md^2}. \label{eq:omega_z} \\
\end{align}

For simplicity.
\autoref{eq:eom_x} and \autoref{eq:eom_y} are coupled, but can write them as one equation by defining

\begin{equation}
f(t) = x(t) + iy(t).
\end{equation}

Taking the double derivative of this, we obtain

\begin{align}
\ddot{f} =& \ddot{x} + i\ddot{y}, \\
=& \omega_0\dot{y} + \frac{1}{2}\omega_z^2x - i\omega_0\dot{x} + i\frac{1}{2}\omega_z^2y, \\
=& \omega_0\left(\dot{y} - i\dot{x}\right) + \frac{1}{2} \omega_z^2\left(x + iy\right), \\
=& -i\omega_0 \dot{f} + \frac{1}{2}\omega_z^2f,
\end{align}

where we have inserted $\ddot{x}$ and $\ddot{y}$ from \autoref{eq:eom_x} and \autoref{eq:eom_y}. The equation for $f$ has analytical solutions on the form (revise: add citation)

\begin{equation}
f(t) = A_+e^{-i(\omega_+t + \phi_+)} + A_-e^{-i(\omega_-t  + \phi_-)},
\label{eq:f_gen_sol}
\end{equation}


where $\phi_{\pm}$ are initial phases,  $A_{\pm}$ are the initial amplitudes,
and the angular frequencies $\omega_{\pm}$ are given by
\begin{equation}
\omega_\pm = \frac{\omega_0 \pm \sqrt{\omega_0^2 - 2\omega_z^2}}{2}
\label{eq:omega_pm}
\end{equation}

The position in the $xy$-plane, contained in $f$, will grow unboundedly when $\omega_{\pm}$ is a
complex number.
For the particles to be trapped, we thus need $\omega_{\pm}$ to be real numbers. We achieve this when
\begin{equation}
\omega_0^2 - 2\omega_z^2 \geq 0.
\end{equation}

Using \autoref{eq:omega_0} and \autoref{eq:omega_z} we can write this as
\begin{equation}
\frac{q}{m} \geq \frac{4V_0}{B_0^2d^2}.
\end{equation}

This indicates that the system is more stable when the particles have a low mass relative to their charge.
Inertial effects will then be less prominent. We see that it is easier to trap particles when the magnetic
field and size of the trap are large relative to the electric field. The increasing strength of the $\bold{B}$-field
will make the particle orbits narrower. Increasing the size reduces the need for a strong $\bold{B}$-field,
as the particles can have a wider orbit without collapsing onto the ring.

Assuming the particle is trapped, we can use \autoref{eq:f_gen_sol} to find upper and lower bounds on the
position of the particle given by its initial position.

We rewrite the positions $x(t) = \mathcal{R}(f(t))$, and $y(t) = \mathcal{I}(f(t))$ using Eulers formula
\begin{align}
x(t) &= A_+ \cos(\omega_+t + \phi_+) + A_- \cos(\omega_-t + \phi_-) \\
y(t) &= -A_- \sin(\omega_- t + \phi_-) -A_+ \sin(\omega_+ t + \phi_+).
\end{align}

The distance from the centre of the trap is $D^2 = x^2 + y^2$. Inserting the above expressions for
$x(t)$ and $y(t)$ we find

\begin{equation}
D^2 = A_+^2 + A_-^2 + 2A_+A_-\cos((\omega_+ - \omega_-)t + \varphi),
\end{equation}

where $\varphi := \phi_+ - \phi_-.$
We find the largest and smallest distance from the centre by maximizing and minimizing the
cosine term. This yields an upper bound $D_{\text{max}} = |A_{+} + A_{-}|$ and a lower bound
$D_{\text{min}} =  \sqrt{A_+^2 + A_-^2 - 2A_+A_-} = |A_+ - A_-|$.

A Penning trap is not limited to trapping only single particles. We need to
consider the Coulomb interactions between multiple particles that will be trapped. The electric field from $N$
particles with charge $q>0$ is the sum of each particle field. We rewrite it as

\begin{equation}
\bold{E} = k_e\sum_{j=1}^Nq\frac{\bold{r} - \bold{r}_j}{|\bold{r} - \bold{r}_j|^3},
\end{equation}

where $k_e$ is the Coulomb constant, $\bold{r_i}$ is the particle position and
\footnote{Except at $\bold{r} = \bold{r}_i$, since this is undefined.}
$r$ is an arbitrary position. The force on each particle from the electric field is $\bold{F}_i = q\bold{E}_i$,
and the charge times the electric field in the particle's position.

We can write the equations of motion for particle number $i$ as 
\begin{align}
  \ddot{x} - \omega_{0}\dot{y} - \frac{1}{2}\omega_{z}^{2}x -\frac{k_eq^2}{m}\sum_{j\neq i}
  \frac{x_i - x_j}{|\bold{r}_i - \bold{r}_j|^3} &= 0, \label{eq:eom_inter_x}\\
  \ddot{y} + \omega_{0}\dot{x} - \frac{1}{2}\omega_{z}^{2}y - \frac{k_eq^2}{m}\sum_{j\neq i}
\frac{y_i - y_j}{|\bold{r}_i - \bold{r}_j|^3}
&= 0, \label{eq:eom_inter_y}\\
  \ddot{z} + \omega_{z}^{2}z -\frac{k_eq^2}{m}\sum_{j\neq i}
  \frac{z_i - z_j}{|\bold{r}_i - \bold{r}_j|^3} &= 0, \label{eq:eom_inter_z}
\end{align}

Each particle can be thought of as a small current. Charges in motion will set up a weak magnetic field
according to Amperes law
\begin{equation}
\nabla \times \bold{B} = \mu_0 \left(\bold{J} + \varepsilon_0 \frac{\partial \bold{E}}{\partial t}\right),
\end{equation}

where $\mu_0$ is the magnetic permeability of the vacuum and $\varepsilon_0$ is the electric permittivity of
the vacuum. We neglect this magnetic field's effect on the particles' motion. If many particles
fly in opposite directions, we expect the contributions from each of them to cancel out.
In comparison, the effect from many like-charged particles will add up. We, therefore, expect this
to be a more prominent effect in our simulations.
revise: $\mu$
smaller than $k_e$? Give reasons for this negligence.

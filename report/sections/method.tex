todo:

-present algorithms FE and RK4 and explain how they are applied in
the Penning trap.\linebreak
-also mention how particle interactions are calculated \linebreak
-probably lots of other stuff
\\



\subsection{Numerical}
Since calculating the analytical solution for $N$ Ca$^{+}$ ions is going to be hard,
we will instead use our machines' power. We want to solve the equations of motion
\ref{eq:eom_inter_x}, \ref{eq:eom_inter_y} and \ref{eq:eom_inter_z} numerically.

\subsubsection{Forward Euler} \label{sec:forward-euler}
We start by deriving the Forward Euler method.
\begin{align}
  \frac{\dd{x}}{\dd{t}} &= f(t, x)\\
  &\Downarrow\nonumber\\
  \frac{x_{i+1}  - x_{i}}{h} + \mathcal{O}(h) &= f_{i}\\
  x_{i+1} &= x_{i} + hf_{i} + \mathcal{O}(h^{2}) \label{eq:12}
\end{align}
Where $h$ is the step size, and $\mathcal{O}(h^{2})$ is the remaining terms.
If we truncate~(\ref{eq:12})
\begin{align}
  x_{i+1} = x_{i} + hf_{i} \label{eq:13}
\end{align}
This then has a truncation error of $\mathcal{O}(h^{2})$ and a global error of $\mathcal{O}(h)$.
However, we have a second-order differential equation. So using the same method but with
\begin{align}
  \frac{\dd^{2}{x}}{\dd{t^{2}}} &= f\left(t, x, \frac{\dd{x}}{\dd{t}}\right).
\end{align}
We will instead have
\begin{align}
  v_{i+1} &= v_{i} + hf_{i}\label{eq:14}\\
  x_{i+1} &= x_{i} + hv_{i} \label{eq:16}
\end{align}
Where it should be noted that you would get the Euler-Cromer method if in (\ref{eq:16}) you used $v_{i+1}$ instead of $v_{i}$.
The function $f$ is, in our case, going to be the particle's acceleration, which we find using Newton's second law. <++> add ref here <++>
So the full algorithm is as follows:

\begin{align}
  \mathbf{a}_{i} &= \frac{\mathbf{F}_{i}}{m}\\
  \mathbf{v}_{i+1} &= \mathbf{v}_{i} + h\mathbf{a}_{i}\\
  \mathbf{x}_{i+1} &= \mathbf{x}_{i} + h\mathbf{v}_{i}\\
\end{align}


Where $\mathbf{a}$ is the 3D acceleration vector, $\mathbf{F}$ is the 3D force vector,
$\mathbf{v}$ is the velocity vector, $\mathbf{x}$ is the position vector, and $m$ is the mass.
We use the same mass since we will only have one type of ion in our trap.

\subsubsection{Runge Kutta 4}
The Runge-Kutta method is an extension of the forward Euler method described in \ref{sec:forward-euler}.
However, we do it four times instead of calculating the slope only once.
The general approach is that for the first slope value $k_{1}$ is found using the \ref{sec:forward-euler}.
Then for the second slope value $k_{2}$ is the slope at the midpoint of the interval from $t$ to $t + h$.
Then we recalculate the midpoint slope using the $k_{2}$ value.
Before we calculate the slope at $t + h$ to get the $k_{4}$.
\begin{align}
  k_{1} &= f\left(t_{n}, x_{n}\right)\\
  k_{2} &= f\left(t_{n} + \frac{h}{2}, x_{n} + h \frac{k_{1}}{2}\right)\\
  k_{3} &= f\left(t_{n} + \frac{h}{2}, x_{n} + h \frac{k_{2}}{d}\right)\\
  k_{4} &= f\left(t_{n} + h, x_{n} + hk_{3}\right)
\end{align}

Before the next step is calculated from
\begin{align}
  x_{n+1} &= x_{n} + \frac{1}{6}\left(k_{1} + 2k_{2} + 2k_{3} + k_{4}\right)h.
\end{align}

Now for our specific case, we need to have one coefficient for the positions and the velocities.
Then we need to calculate all the particles' coefficients before moving on to the next coefficient.


So our specific algorithm will have this shape:
\begin{lstlisting}[language=C]
// Use the forward Euler method to move all particles one step
evolve_forward_Euler(i, t);

// For all the particles
  l1[n] = h * F(n, t);
  k1[n] = h * v_[n];

// For all the particles
  position[n+1] = position[n] + k1[n]/2.;
  velocity[n+1] = velocity[n] + l1[n]/2.;

// For all particles
  l2[n] = h * F(n, t + 0.5*h);
  k2[n] = h * velocity[n+1];

// For all particles
  position[n+1] = position[n] + k2[n] / 2.;
  velocity[n+1] = velocity[n] + l2[n] / 2.;

// For all particles
  l3[n] = h * F(n, t + 0.5*h);
  k3[n] = h * velocity[n+1];

// For all particles
  position[n+1] = position[n] + k3[n];
  velocity[n+1] = velocity[n] + l3[n];

// For all particles
  l4[n] = h * F(n, t + h);
  k4[n] = h * velocity[n+1];

// For all particles
  position[n+1] = position[n] + (k1[n] + 2.*k2[n] + 2.*k3[n] + k4[n]) / 6.;
  velocity[n+1] = velocity[n] + (l1[n] + 2.*l2[n] + 2.*l3[n] + l4[n]) / 6.;
\end{lstlisting}

Where \lstinline{F(n, t)} is the force on the \lstinline{n} particle at time $t$.
